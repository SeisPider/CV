% AGU style: https://publications.agu.org/agu-grammar-and-style-guide/
\newcommand{\Revision}{\textit{under revision}}
\newcommand{\CS}{*} % corresponding author
\newcommand{\CF}{\textsuperscript{\#}} % co-first author


\section*{Peer-reviewed Publications}
%\CS corresponding author, \CF co-first author.

\begin{etaremune}
\item
    Cheng, S. H., \Xiao, Wu, J. P., Wang, W. L., Sun, L., Wang, X. X. \& Wen, L. X.(2021).
    Crustal Thickness and Vp/Vs Variations Beneath the Continental China Revealed by Receiver Function Analysis.
    \textit{Geophysical Journal International}, \textit{228}(3), 1731-1749. \DOI{10.1093/gji/ggab022} 
\item
    \Xiao, Cheng, S. H., Wu, J. P., Wang, W. L., Sun, L., Wang, X. X. \& Wen, L. X.(2021).
    Shallow seismic structure beneath China revealed by P wave polarization, Rayleigh wave ellipticity and receiver function.
    \textit{Geophysical Journal International}, \textit{225}(2), 998-1019. \DOI{10.1093/gji/ggab433}
\item
    Chen, Z. Luo, J.,  \Xiao, \& Sun, F.(2017).
    Assessment of COSMIC radio occultation water vapor profile.
    \textit{Journal of National University of Defense Technology}, \textit{39}(3), 201--206.
\end{etaremune}

\section*{Papers Submitted/Under Review}
\begin{etaremune}
\item
    Yao, J. Y., Wu, S. C., Li, T. J., \Xiao, Hubbard, J., Wang, Y., He, Y. M., Thant, M., \& Tong, P. (2022).
    Imaging the upper 10 km crustal shear-wave velocity structure of central Myanmar via a joint inversion of body-wave polarizations and receiver functions.
    \textit{Seismological Research Letter} [\textit{Under Review}]
\item
    \Xiao, Sun, L., Wang, X. X. \& Wen, L. X.(2022).
    Simultaneous inversion for surface wave phase velocity and earthquake centroid parameters: methodology and application.
    \textit{Journal of Geophysical Research: Solid Earth} [\textit{Under Review}]
\item
    Mao, S., Cheng, S. H., \Xiao, Wu, J. P., Wang, W. L., Sun, L., Wang, X. X., \& Wen, L. X.(2022).
    A three-dimensional adaptive receiver function migration method: (II) application to southeastern Tibetan plateau.
    \textit{Journal of Geophysical Research: Solid Earth} [\textit{Submitted}]
\end{etaremune}

\section*{Papers in Preparation}
\begin{etaremune}
\item
    \Xiao, Cheng, S. H., Wu, J. P., Wang, W. L., Sun, L., Wang, X. X. \& Wen, L. X.(2023).
    Seismic structure of the crust and uppermost mantle beneath China from various seismic constraints.
\end{etaremune}

