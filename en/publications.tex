% AGU style: https://publications.agu.org/agu-grammar-and-style-guide/
\newcommand{\Revision}{\textit{under revision}}
\newcommand{\CS}{*} % corresponding author
\newcommand{\CF}{\textsuperscript{\#}} % co-first author


%%%%%%%%%%%%%%%%%%%%%%%%%%%%%%%%%%%%%%%%%%%%%%%%%%%%%%%%%%%%%%%%%%%%%%%%%%%%%%%
\section{Peer-reviewed Publications}
% , \CF co-first author.
\CS corresponding author

\begin{EntriesTableExtra}
  \Year{2022} & 
  \Me\CS, \sunl, \wxx, \& \wlx \ 
  Simultaneous inversion for surface wave phase velocity and earthquake centroid parameters: methodology and application.
  \emph{Journal of Geophysical Research: Solid Earth}.
  \DOI{10.1029/2022JB024018}.
  & 
  \GitHub{seispider/SimInvVelRelocate}
  \\
   ~ & 
   \yjy, Wu, S., Li, T., Bai, Y., \Me, Hubbard, J., Wang, Y., He, Y., Thant, M., \& Tong, P. \ 
   Imaging the upper 10 km crustal shear-wave velocity structure of central Myanmar via a joint inversion of body-wave polarizations and receiver functions.
   \emph{Seismological Research Letter}.
   \DOI{10.1785/0220210292}.
  \\
   \Year{2021} & 
   \csh, \Me, \wjp, \wwl, \sunl, \wxx, \& \wlx \ 
   Crustal Thickness and Vp/Vs Variations Beneath the Continental China Revealed by Receiver Function Analysis.
   \emph{Geophysical Journal International}.
   \DOI{10.1093/gji/ggab022}
   \\
   ~ & 
   \Me\CS, \csh, \wjp, \wwl, \sunl, \wxx, \& \wlx \ 
   Shallow seismic structure beneath China revealed by P wave polarization, Rayleigh wave ellipticity and receiver function.
   \emph{Geophysical Journal International}.
   \DOI{10.1093/gji/ggab433}.
   &
   \GitHub{seispider/ShallowSeismicImaging}
\end{EntriesTableExtra}

\section{Papers Submitted/Under Review}

\begin{EntriesTableExtra}
    \Year{2022} &
    \msz, \csh, \Me, \wjp, \wwl, \sunl, \wxx, \& \wlx \ 
    \emph{Journal of Geophysical Research: Solid Earth}
    [\textit{Submitted}]
\end{EntriesTableExtra}

\section{Papers in Preparation}
\begin{EntriesTableExtra}
    \Year{on} &
    \Me, \csh, \wjp, \wwl, \sunl, \wxx, \& \wlx \
    A Preliminary Crustal Shear Wave Velocity Model for the continental China.
\end{EntriesTableExtra}

