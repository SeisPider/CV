%%%%%%%%%%%%%%%%%%%%%%%%%%%%%%%%%%%%%%%%%%%%%%%%%%%%%%%%%%%%%%%%%%%%%%%%%%%%%%%
\section{Presentations}

\subsection{Invited \& Keynotes}
\begin{EntriesTableExtra}
  \Year{2018} &
  \Me, \csh, \& \wlx \ 
  Shallow shear wave structure beneath China revealed by rayleigh wave ellipticity and receiver function.\ \textit{USTC},   Dec. 25, 2018. \textbf{[Student Seminar]}
\end{EntriesTableExtra}


\subsection{Other Presentations}
\begin{EntriesTableExtra}
  \Year{2022} &
  \Me, \sunl, \wxx, \& \wlx \ 
  Simultaneous inversion for surface wave phase velocity and earthquake centroid parameters: methodology and application.
  \emph{AGU 2022}, Chicago, IL, USA and \emph{CGU 2022}, Online, CHN
  \\
  \Year{2019} &
  Xu, Y., \sunl, Hao, J., Lu, Z., \Me, \& \wlx \ 
  Source properties of 17 June 2019 Changning earthquake (Mw 6.2), China and its aftershocks.
  \emph{AGU 2019}, San Francisco, CA, USA.
  \\
  ~ &
  Zhu, J., Lu, Z., Xu, Y., \Me, \wxx, \& \wlx \ 
  Temperature-related Martian seismic events observed by InSight.
  \emph{AGU 2019}, San Francisco, CA, USA.
  \\
  ~ &
  \msz, \csh, \Me, \wjp, \& \wlx \ 
  A three-dimensional receiver function migration method imaging the crustal structure in Sichuan-Yunnan Region, Southwest China.
  \emph{AGU 2019}, San Francisco, CA, USA.
  \\
  ~ &
  Lu, Z., \Me, \csh, \wxx, Zhu, J., \& \wlx \ 
  Shallow Martian Seismic Velocity Structure Inferred from InSight's Seismic Signals Produced by Air Pressure Variations.
  \emph{AGU 2019}, San Francisco, CA, USA.
  \\
  ~ &
  \Me, \csh, \& \wlx \ 
  A Preliminary Crustal Shear Wave Velocity Model for the continental China. 
  \emph{AGU 2019}, San Francisco, CA, USA.
  & 
  \Slides{github.com/SeisPider/Posters/tree/master/AGU2019}
  \\
  \Year{2018} &
  \Me, \csh, \& \wlx \ 
  Shallow seismic structure beneath China revealed by body-wave polarization and Rayleigh-wave ellipticity.
  \emph{AGU 2018}, Washington, DC, USA.
  & 
  \Slides{github.com/SeisPider/Posters/tree/master/AGU2018}
  \\
  \Year{2017} &
  \Me, \& \wlx \ 
  3D Crust and Uppermost Mantle Structure beneath Tian Shan Region from ambient noise and earthquake surface waves.
  \emph{AGU 2017}, New Orleans, LA, USA.
  & 
  \Slides{github.com/SeisPider/Posters/tree/master/AGU2017}
\end{EntriesTableExtra}