\section*{Meeting Abstracts}
\begin{etaremune}
\item
    Xu Y., Sun L., Hao J., Lu Z., \Xiao \& Wen, L. (2019).
    Source properties of 17 June 2019 Changning earthquake (Mw 6.2), China and its aftershocks.
    Abstract S11G-0437 presented at 2019 AGU Fall Meeting, San Francisco, CA, USA.
\item
    Zhu J., Lu Z., Xu Y., \Xiao, Wang X. \& Wen, L. (2019).
    Temperature-related Martian seismic events observed by InSight.
    Abstract DI51B-0025 presented at 2019 AGU Fall Meeting, San Francisco, CA, USA.
\item
    Mao S., Cheng S., \Xiao, Wu J. \& Wen, L. (2019).
    A three-dimensional receiver function migration method imaging the crustal structure in Sichuan-Yunnan Region, Southwest China. 
    Abstract S21D-0534 presented at 2019 AGU Fall Meeting, San Francisco, CA, USA.
\item
    Lu Z., \Xiao, Cheng S., Wang X., Zhu J. \& Wen, L. (2019).
    Shallow Martian Seismic Velocity Structure Inferred from InSight's Seismic Signals Produced by Air Pressure Variations. 
    Abstract DI51A-0015 presented at 2019 AGU Fall Meeting, San Francisco, CA, USA.
\item
    \Xiao, Cheng S.\& Wen, L. (2019).
    A Preliminary Crustal Shear Wave Velocity Model for the continental China. 
    Abstract S11D-0376 presented at 2019 AGU Fall Meeting, San Francisco, CA, USA.

\item
    \Xiao, Cheng S.\& Wen, L. (2018).
    Shallow seismic structure beneath China revealed by body-wave polarization and Rayleigh-wave ellipticity. 
    Abstract S23C-0530 presented at 2018 AGU Fall Meeting, Washington, DC, USA.
\item
    \Xiao, \& Wen, L. (2017).
    3D Crust and Uppermost Mantle Structure beneath Tian Shan Region from ambient noise and earthquake surface waves. 
    Abstract S51D-062 presented at 2017 AGU Fall Meeting, New Orleans, LA, USA.
\end{etaremune}
